\documentclass{article}
\setlength{\parindent}{0pt}

\usepackage{url}
\usepackage{hyperref}
\begin{document}

%\newenvironment{myoutput} { \begin{center}\framebox[1.05\width]\{\begin{minipage}[c]{0.8\textwidth} \small\{ \begin{verbatim} } { \end{verbatim}\} \end{minipage} \} \end{center} }

% ---------------------------------------------------------------------
% ---------------------------------------------------------------------


\section{Before You Start}

Running a Mix takes some commitment. You will need to set aside some
CPU cycles (idealy on a separate machine) and probably a lot of
bandwidth. Don't try to run a Mix in your network if you don't have
the authority or permission to do so. 

% ---------------------------------------------------------------------

\subsection{Some Mix Basics}

In this document we will assume that you know the basics of what a Mix
is and what a Mix-Cascade is. We also assume that you know a little
about cryptography and the terms ``public key'' and ``private key''
are not all Greek to you.

Please check out \url{http://www.anon-online.com/} for details on how
a Mix works and what a Cascade is.

TO-DO: add some basics here.

% ---------------------------------------------------------------------

\subsection{Planning a Cascade}

Currently getting a Mix known to the JAP users involves some
coordination with us and other Mix operators.

You will need the public keys of the other Mixes and they will need to
get your Mix's public key (don't worry we will get to the creation of
your key later).

We will send you the public keys of the other Mixes as attachments
to an email. (TO-DO: Downloading public keys from InfoService)

We will also help you get in touch with the other operators.
(TO-DO: InfoService Operator-fields with contact information.)

Our email address is \url{jap@inf.tu-dresden.de} for the time being.
\footnote{This is also our support address for our users so we may install a
separate contact address for operators soon.}

\subsection{Keys and Trust}

Due to the inherent insecurity of the Internet messages sent to us and
received from us should be cryptographically signed. There is no need to
encrypt messages since we will not exchange any secret information. In
fact it is better not to encrypt emails in this context since signed
emails will provide some transparency and reassurance in case of doubt.

Our PGP key can be found at \url{http://anon.inf.tu-dresden.de/jappgp.asc}
and the key's fingerprint is ``B965 99E4 05EB 4202 895C 27DC D022 60C9 73EE 1DD1''

If you don't trust this document itself you should confirm the key's
fingerprint by some other means e.g. calling us on the phone.

If you simply send us your PGP or GPG key in an unsecured way (e.g.
email) we will also not trust it until it has been confirmed by some
other means (like a phone call).

% ---------------------------------------------------------------------

\subsection{Voluntary Commitment}

Mix operators are given a lot of trust by their users. Therefore a Mix
operator should sign a commitment that states the way he is going to
operate the Mix.

This commitment is meant not only for the end users, but also for the
other Mix operators.

Currently there is only a German version of it at
\url{http://www.anon-online.com/Selbstverpflichtung.html} but we are
working on an English version of that document too.

Please sign this document and send us the signed copy. Sending us a paper
version would be ideal (for the legal people), as well as an
electronic version (for the users). Check out the above mentioned URL
for an example of the electronically signed version.

% ---------------------------------------------------------------------
% ---------------------------------------------------------------------

\section{System Installation}
\label{sys_install}

We will try to guide you through the installation of a basic Linux
system. No bells or whistles. Just the plain basic system. This will,
however, be a rather sketchy description and by no means a replacement
for reading the installation manual of your Linux distribution.

A word of caution: Using the newest hottest version of any software
will most definitely not result in the most secure system. New
features tend to introduce new bugs and new security problems.
Therefore, if facing the choice between a stable and a developers
version, it is better to stick to the stable one (provided that
security fixes are available when needed).

% ---------------------------------------------------------------------

\subsection{Linux - choosing your flavor}

What ``basic Linux system'' means depends on what you want to do with
the system. But first of all you may need to decide which distribution
of Linux to run.

If you are an experienced Unix operator we would strongly recommend a
Debian system. It is secure out-of-the-box and has the fastest
security update service. However if has a much steeper learning curve
when compared to other, more desktop oriented, Linux distributions.

If you are not familiar with Debian and you prefer graphical
configuration tools you might want to try Red Hat or Suse instead.

For operating a Mix you just need a text console. You don't even need
X-Windows or any other graphical display. 

For configuring the Mix however there is a graphical Java
Applet/Application. That program can be run on your Mix computer or on
another computer (even a Windows system can do it). Running it on a
different computer may introduce some security problem since you will
need to handle the Mix's secret key. Installing X-Windows and a Java
runtime environment on the Mix will require much more space and a lot
more software on the Mix itself. The choice is up to you.

For the rest of this document we will assume that you want to run the
Configuration Tool on your Mix computer and therefore you will need an
X-Windows system and a Java runtime engine. 

% ---------------------------------------------------------------------

\subsubsection{Debian 3.0 (Woody)}

Install Debian Woody. There is a very detailed installation manual
online at \url{http://www.debian.org/releases/stable/installmanual}
but simply following the instructions during installation will also
get you there.

Make sure you include the ``non-US'' and ``non-free'' packages when
asked during the installation. Otherwise you will probably have some
trouble with the OpenSSL packages.

When asked for additional apt-sources add
\url{ftp://ftp.informatik.hu-berlin.de/pub/Java/Linux/debian} to get
Java pre-packaged for Debian.

When asked by the ``Task Installer'' during the installation which
tasks to install, select ``X window system'' from the end-user section
and ``C and C++'' from the Development section.

After completing the installation, log in as root and install these
additional packages: 

\begin{itemize}
\item libxerces1-dev
\item libxerces1
\item openssl
\item ssh 
\item j2re1.3
\item j2sdk1.3
\end{itemize}

You can do it by running \verb|dselect| or by using 
\verb|apt-get install <package-name>|. If the dselect program complains about a
missing j2se-common package you will have to press ``Q'' to ignore
this.

\subsubsection{Suse Linux 8.2}

During installation you will be asked which software to install.

First go to the selections and deactivate ``Office Applications''.
This will save a lot of space and time.  Then switch the filter from
``Selections'' to ``Package groups''.

\parbox{\textwidth}{
  In the tree view on the left, now select the packages ``openssl-devel'', ``Xerces-c-devel''
and ``mozilla''
  \begin{itemize}
  \item[$\vdash$] Development
    \begin{itemize}
    \item[$\vdash$] Libraries
      \begin{itemize}
      \item[$\vdash$] C and C++
        
        
        \begin{itemize}
        \item[$\Rightarrow$] openssl-devel
          
        \item[$\Rightarrow$] Xerces-c-devel
        \end{itemize}
        
      \end{itemize}
    \end{itemize}
  \end{itemize}
}\par

TO-DO: Commercial Suse contains the Java package .. but where.. or is
it installed by default?

\begin{center}
\framebox[1.05\width]{
  \begin{minipage}[c]{0.8\textwidth}
    If you don't have the full commercial version of Suse you will need to
    download the Java runtime environment after completing the system
    installation. It can be found at\\
    \url{http://java.sun.com/j2se/1.3.1/download.html}.\\
    Select the ``Linux RPM in self-extracting file'' of the JRE.
    Currently it is named \\
    \texttt{j2re-1\_3\_1\_08-linux-i586.rpm.bin}.\\
    After downloading the file run\\
    \texttt{bash j2re-1\_3\_1\_08-linux-i586.rpm.bin}\\
    to extract it. This will produce a file named\\
    \texttt{jre-1.3.1\_08.i586.rpm}.\\
    To finally install this package, you will have to become root again. 
    Open a console and run \\
    \texttt{su -} \\
    \texttt{rpm -iv jre-1.3.1\_08.i586.rpm}\\
    The disadvantage of this is that you will have to type\\
    \texttt{/usr/java/jre1.3.1\_08/bin/java}\\
    instead of simply \texttt{java} later when we need the Java 
    environment. 
  \end{minipage}
}
\end{center}

TO-DO: \\
Deactivate these services:\\
portmap TCP 111\\
X TCP 6000\\

\subsubsection{Red Hat 9}

Install Red Hat with the default Workstation settings. To save some
time and space you might want to deselect ``mrproject'' and
``openoffice'' from the ``Office/Productivity'' group.

As with the free Suse version you have to download the Java runtime
environment if you want to use the MixConfig tool. Take a look at the
previous section for details.

The precompiled binary of the Mix works ok on RedHat9
\footnote{Earlier versions had some kernel-thread issues.} but if you
want to compile your own mix, you will have to download the xerces-c libraries and 
compile and install them first.

Taking the precompiled xerces-c binaries for RedHat 7.2 will fail with\\
\texttt{configure: error: Xerces-C Lib NOT found - please use --with-xml-lib}\\
because of incompatible binary interfaces.

You can download the sources of xerces-c from
\url{http://xml.apache.org/xerces-c/} and follow the ``Install'' and
``Build'' instructions.

Last time we checked version 2.3.0 was the current version and didn't
work with proxytest.\footnote{They changed the API again without
  warning. proxytest 0.2.1 works with xerces-c 2.3.0 now but the next
  xerces-c version might break things again.}

Rather try the last 1.x release ... 1.7 works without a problem.

Download the sources and install according to the included Documentation 
( accessible by the Readme.html file )


After runing ``gmake'' as a final build step for xerces-c you don't
need to install the libraries anywhere. 
Make sure that there is a link named libxerces-c.so in the librarie directory by this commands:

\begin{verbatim}
$ cd $XERCESCROOT/lib
$ ln -is libxerces-c1_7_0.so libxerces-c.so
\end{verbatim}


The rest is the same as described in \ref{compilemix}.

TO-DO: \\
Deactivate these services:\\
Install these packages:\\

% ---------------------------------------------------------------------

\subsection{Other Linux Flavors}

If you want to install a Mix on a different Linux install the
following packages using your package management system or from sources.

\begin{itemize}
\item Xerces-c libraries and development files. (version 1.5 or above should be OK)

\item OpenSSL libraries and development files. (0.9.6a or above)

\item Java Runtime Environment (Version 1.3 or above)
\end{itemize}

% ---------------------------------------------------------------------

\subsection{tools you will also need}

\begin{itemize}
\item{A Web Browser}. ``Lynx'' will do but when X-Windows is installed most
people will want to install Netscape, Mozilla, Opera or Konqueror. Any
one of those will do.

\item{An email client}. Again ``mutt'' will perfectly suffice but you can
also use a graphical one like KMail or Evolution. Preferably one that
has builtin support for PGP or GPG. (If you ask about Outlook Express
you will be sent back home.\footnote{Do not pass Go, do not collect
  \$200!})

\item{Crypto-software}. PGP (Pretty Good Privacy) or GPG (Gnu Privacy
  Guard) for signing emails and checking signed emails.

\item{Ssh} (secure shell) for remote administration of your Mix.
\end{itemize}

\subsection{network connectivity test}

TO-DO

% ---------------------------------------------------------------------
% ---------------------------------------------------------------------


\section{Mix Installation}

In this section we will go through the steps to get the Mix program ready
to run on your system. 

% ---------------------------------------------------------------------

\subsection{Create a User}

Creating a separate user for running the Mix software is a good idea.
It will keep the Mix software out of your way (and you out of its
way). For the sake of simplicity call that new user ``mix''.

Most systems will have a graphical tool for managing users but simply
running \verb|adduser mix| as root will do just the same.

% ---------------------------------------------------------------------

\subsection{Download Mix Software}

Login as the user ``mix''.

Next you need the Mix software. The binary (the
executable file) is called \verb|proxytest| and you can download a
precompiled version of it or compile it on your own.

If you are in a hurry you may want to try the precompiled version. 

If you want to do it the \emph{right} way (or the precompiled versions
doesn't work on your system) you have to download the sources and compile it
yourself.

% ---------------------------------------------------------------------

\subsubsection{Precompiled Mix Software}
\label{precompiled}
Start your browser and go to

\url{http://anon.inf.tu-dresden.de/develop/aktversions_en.html}

Download the \verb|proxytest.tgz| archive and save it to your home directory (\texttt{/home/mix}).

Open a shell and extract the archive with \verb|tar xzvf proxytest.tgz|

This will extract the precompiled binary named \verb|proxytest|

To check if the precompiled version runs on your system you can try to run
it in the JAP\-Mode\footnote{This is basically a JAP replacement mode. It
makes the Mix software act like a proxy for your web browser.}\\

\verb|./proxytest -j -a -p 4001 -n mix.inf.tu-dresden.de:6544|

The output should look something like this:
\begin{verbatim}
[2003/06/16-18:01:30, info    ] Anon proxy started!
[2003/06/16-18:01:30, info    ] Version: 00.01.67
[2003/06/16-18:01:30, info    ] Using: OpenSSL 0.9.6c 21 dec 2001!
[2003/06/16-18:01:30, info    ] Starting MIX...
[2003/06/16-18:01:30, info    ] Try connecting to next Mix...
[2003/06/16-18:01:32, info    ]  connected!
[2003/06/16-18:01:32, info    ] Chain-Length: 2
\end{verbatim}

The version numbers may vary. By pressing \verb|CTRL-C| you can stop it again.

If you don't get connected within 30 seconds you may need to try a different
address behind the ''-n'' option. With your browser, go to
\url{http://anon.inf.tu-dresden.de/status.php?lang=en} and click on one of
the available cascade's name. The resulting page will list under ''General
Information'' an IP and a Port to try instead of the ones listed above.

If proxytest dies with an error message like
this one

\verb|TODO|

then you will need to compile your own version.

% ---------------------------------------------------------------------

\subsubsection{Compiling the Mix Software from Sources}
\label{compilemix}

Make sure that you have the development packages installed as mentioned
under Section \ref{sys_install}.

Start your browser and go to
\url{http://anon.inf.tu-dresden.de/develop/sources_en.html}

Download the \verb|mix.src.tgz| archive and save it to your home directory.
Open a shell and extract the archive with \verb|tar xzvf mix.src.tgz| \\
This will create a directory named \verb|proxytest|. Please rename it by
typing \verb|mv proxytest proxytestdir| 

Change into that directory and run \verb|./configure| 

The output will be something like 
\begin{verbatim}
checking for a BSD-compatible install... /usr/bin/install -c
checking whether build environment is sane... yes
checking for gawk... no
checking for mawk... mawk
\end{verbatim}
\vdots
\begin{verbatim}
checking for OpenSSL directory... (system)
checking for dom/DOM.hpp... yes
Xerces-C includes found
XercesC-Lib found
configure: creating ./config.status
config.status: creating Makefile
config.status: creating popt/Makefile
config.status: creating xml/Makefile
config.status: creating aes/Makefile
config.status: creating trio/Makefile
config.status: creating config.h
config.status: executing depfiles commands
\end{verbatim}

There is a new \verb|configure| switch in recent versions (>00.02.x)
of proxytest that allows for a deanonymization of certain connections.
The switch is \verb|--with-crime-detection|. 

Normally it would require all mix operators to work together to
deanonymize a connection. Due to a design flaw in the current
proxytest versions however, it is possible to circumvent older
versions of the middle mix. Newer versions (>=00.01.69) contain a fix
for the middle mix but the fundamental weakness will only be fixed by
a new protocol. Probably in Version 00.03.x.

For details look for \verb|LOG_CRIME| and \verb|ALLOWED_FLAGS|
switches in the source code.


If configure didn't completely run through you can check the config.log file for 
errors (e.g. using \verb|less config.log|). 
Toward the end of that file you may find a clue as to what went wrong. 
Look for a line that contains the error message that you got from configure. 
Like this one:
\begin{verbatim}
configure:5328: error: Xerces-C Lib NOT found - please use --with-xml-lib
\end{verbatim}
Above this error message you will find the test that caused the error and
possibly a more detailed error message.

A word of warning about the C/C++ compiler version: On some systems
there has been some trouble with gcc versions 3.x and above. Compiling
works but linking to the libraries fails. This is almost exclusively
due to the fact that the libraries had been compiled with a different
compiler version and the binary interfaces between applications and
libraries have changed since than. When in doubt go with the
default version that your Linux distribution installs.

After successfully running configure you can compile the Mix with a simple 
\verb|make| command.

This will create a file named \verb|proxytest|. Please move it to the
parent directory by typing \verb|mv proxytest ../|

Test your proxytest binary as mentioned in Section \ref{precompiled} and continue 
with the configuration below.

% ---------------------------------------------------------------------
% ---------------------------------------------------------------------

\section{Configuration} 


The Mix software needs some data to make it work as a Mix in a
cascade. This configuration data is stored in an XML file and you can
create it with the Java application ``MixConfig''.

\subsection{Getting MixConfig}

Stay logged in as ``mix''. Download the program from
\url{http://anon.inf.tu-dresden.de/develop/MixConfig.html} with your
web browser. It also works as a Java Applet but as with all applets,
input and output operations are limited to the system's clip board.
You can't read or write files from an Applet and therefore we recommend
downloading the \verb|MixConfig.jar| file and saving it to you home
directory.

\subsection{MixConfig Step-by-Step}

Open a shell and run \verb|java -jar MixConfig.jar|

The configuration tool has four main areas to complete.
You can complete them in any order you want.

Through the ``File'' menu you can start a new configuration, save it
to a file and later reload your configuration file to make changes.

\subsubsection{General}

\begin{description}

\item{Mix Type}: Choose the type of Mix you want to run. (This should be agreed upon
  with the other operators in your cascade.)

\item{Cascade Name}: If you are the first Mix you will need to set the name for the whole
  cascade.  A format like ``Mixlocation1 - Mixlocation2 - Mixlocation3''
  or ``Operator1 - Operator2'' would be nice. 
  
  Example: \verb|New York - Berlin - Dresden| 

\item{Mix Name}: Just choose one. It will be visible to the JAP users
  if they check the detailed description for a cascade.
  
  Examples: \verb|Mix at the Data Commisionars Office| or \verb|Berlin1|

\item{Mix ID}: Just push the ``Generate'' button.
  
\item{Set User ID on Execution} will make the Mix change to a
  different User ID when started as root. We will assume that you run
  the Mix as a non-root user anyway so you can leave this option
  switched off.
%, or enter \texttt{mix} 

  Example: \texttt{nobody}
  
\item{Set Number of File Descriptors} artificially limits how many
  concurrent connections your Mix can handle. For the first Mix it is
  the number of users it supports and for the last Mix it is the
  number of concurrent requests it may send to the proxy. (Middle
  mixes only need one connection to the previous Mix and one to the
  next.)
  
\end{description}

The next two options control how much you see of the Mix's work.

During the installation it would be best to switch ``Run as Daemon''
off and ``Enable Logging'' to ``console''.

After successfully testing the Mix in a cascade you should switch on
``Run as Daemon'' and switch off the logging or set it to ``syslog''.

\begin{description}

\item{Run as Daemon}: Set this option to make the mix detach itself from the 
  starting tty like all other Unix daemons.
  
\item{Enable Logging} will make the Mix log some errors warnings and
  debug information. 
  
  For debugging reasons the first Mix will currently also output
  starting and ending connections.  But don't worry.  Before printing
  the IP number of the incoming connection it will be encrypted
  with a random number (generated during the Mix's start). So you can
  at most log that you get connections form \emph{different} IP
  addresses, not from \emph{which} addresses. As soon as you stop the
  mix program that random number is forgotten.

  The last Mix will currently output some byte and packet counters.
  
  For the Unix gurus: All debug output here is subject to change in
  further versions of the Mix software, so don't write wrapper scripts
  that count on it.

\end{description}


\subsubsection{Network}

Here you will need some data that you get from the other Mix operators.
Like theirs Mix's IP addresses and port numbers.

\begin{description}
  
\item{Incoming}: Click on ``Add'' to define how your Mix will receive
  incoming connections. For a First Mix this will be the IP address
  and port where the JAP users connect to. For the other Mix types
  this is the address that their predecessor Mix in the cascade
  connects to. (TO-DO: Currently a Mix also needs the hostname of
  its predecessor, but this will be changed soon.)
  
  Select ``TCP'', ``Raw'' (don't worry the data is encrypted anyway)
  and enter at least your IP address and the port you agreed upon with
  your predecessor Mix or us.
  
\item{Outgoing}: Click on ``Add'' to define how your Mix will connect
  to the next Mix or the final proxy. For a First Mix or Middle Mix
  this will be the IP address and port where they connect to the next
  Mix. For the Last Mix this is the address of a proxy. 
  
  First or Middle Mixes normally select ``TCP'', ``Raw'' and enter at
  least the IP address and the port you agreed upon with your
  successor Mix. 
  
  Last Mixes will enter the proxy's type address and port. \\
  Example: A local squid proxy will be connected as ``HTTP'', ``TCP''
  and ``RAW'' at IP address ``127.0.0.1'' and port ``3128''.

\item{InfoService} describes the InfoService that your Mix will register with.
Currently this is:

\begin{description}
\item{Host}: \verb|infoservice.inf.tu-dresden.de| 
\item{IP}: \verb|141.76.46.91| 
\item{Port}: \verb|6543| 
\end{description}  

The InfoService propagates available Mix cascades to the JAP users. To
get listed with it you will have to get in touch with us and send us
your Mix's public key.

If you want to run a cascade for testing purposes without registering
it to the InfoService you will need to enter the IP address and port
number of the first Mix manually in JAP (or you can use proxytest in
JAP-Mode).

\end{description}

\subsubsection{Certificates}

\begin{description}
\item{Own}: Click on ``Create a New One''. This will take a while
  since it includes the creation of your public and private key.
  Select the time frame (one year starting from now will be OK). Don't
  enter a password, unless your Mix computer is connected to a
  UPS (Uninterruptible Power Supply). If you enter a password here you
  will need to enter it whenever the Mix is restarted. In case of
  power failure it Mix will be unable to start up unattended.
  
  Then click on Export and choose the ``Base64 encoded Public X.509
  Certificate'' Enter a filename like ``mymix'' and the program will
  append \verb|.b64.cer| to the filename.
  
  This file now contains your public key that you need to send to
  us and the other Mix operators that you connect to.
  
\item{Previous and/or Next Mix Certificate}: Click on Import and
  select the public key file(s) that you received from the other
  Operators.

\end{description}

\subsubsection{Network}

\begin{description}
\item{Location}: Where is your Mix. This may be interesting when it
  comes to questions of jurisdiction.
\item{Operator}: Who is running the Mix and how can I get in touch with him.
  \begin{description}
  \item{Organization}: This should contain either a Person's name or the
    Organization that runs the Mix. It can also be a X.500 DN
    (Distinguished Name) if you are familiar with that.
  \item{URL}: A URL pointing to further information about the Operator
    and a means of contact to the person actually running the Mix.
  \end{description}
\end{description}

After completing those forms you should select ``Check'' from the
``File'' menu to see if everything is syntactically OK. Save your
configuration in a file named \verb|config.xml| and close the
MixConfig program.

\subsection{Make Your Public Key Known}

Send your Mix's public key (remember that mymix.b64.cer file) to us
and the other Mix operators of your cascade. Preferably by a secure
means. Either sign your e-mail with PGP/GPG or be prepared to verify
it by some other means.

\subsection{First Start}

Try to start your mix by running \verb|./proxytest -c config.xml|

The output should look something like this for a First Mix:

\begin{verbatim}
[2003/06/17-22:43:27, info    ] Anon proxy started!
[2003/06/17-22:43:27, info    ] Version: 00.01.60
[2003/06/17-22:43:27, info    ] Using: OpenSSL 0.9.6c 21 dec 2001!
[2003/06/17-22:43:27, info    ] I am the First MIX..
[2003/06/17-22:43:27, info    ] Starting MIX...
[2003/06/17-22:43:27, debug   ] Starting FirstMix InitOnce
[2003/06/17-22:43:27, info    ] SOCKET Option SENDLOWWAT not set!
[2003/06/17-22:43:27, info    ] MUXOUT-SOCKET RecvBuffSize: 131070
[2003/06/17-22:43:27, info    ] MUXOUT-SOCKET SendBuffSize: 131070
[2003/06/17-22:43:27, info    ] MUXOUT-SOCKET SendLowWatSize: 1
\end{verbatim}

And like this for a Last Mix:

\begin{verbatim}
[2003/06/18-03:41:05, info    ] Anon proxy started!
[2003/06/18-03:41:05, info    ] Version: 00.01.60
[2003/06/18-03:41:05, info    ] Using: OpenSSL 0.9.6c 21 dec 2001!
[2003/06/18-03:41:05, info    ] Starting MIX...
[2003/06/18-03:41:05, debug   ] This mix will use the following proxies:
[2003/06/18-03:41:05, debug   ] 1. Proxy's Address: 160.45.111.18:3128
[2003/06/18-03:41:06, info    ] Waiting for Connection from previous Mix...
\end{verbatim}


Eventually your Mix should get the connection to its neighbor Mixes and it should go on like this:

\begin{verbatim}
[2003/06/17-22:43:47, info    ]  connected!
[2003/06/17-22:43:47, info    ] Socket option TCP-KEEP-ALIVE returned an error - so not set!
[2003/06/17-22:43:47, critical] Received Key Info lenght 572
[2003/06/17-22:43:48, critical] Received Key Info...
[2003/06/17-22:43:48, debug   ] Get KeyInfo (foolowing line)
[2003/06/17-22:43:48, debug   ] <?xml version=''1.0'' encoding=''UTF-8''?>
<Mixes count=''1''><Mix id=''160.45.111.180X1.6EBD600000400P-8959000''>
<RSAKeyValue><Modulus>tF0prcug5eWsA1lk4RDhI7Dv2E0fI/IzyLOjEjDyHnDLDKDTpNvin/4d2CFX5h33
wNYZle9gtRLJCzFMLeK0sAholzxbJtmDVmvgRIBO15BhuhzGWclClHGrj5VfNnWe
Jc6Y966+ywBgUFuBhb7j3mMQFmc05HfgVbIfZ+aB8vE=
</Modulus><Exponent>AQAB
</Exponent></RSAKeyValue><Signature><SignedInfo><Reference URI="><DigestValue>ET8S0RPHcuSwoS3LoBhLWnDLneE=
</DigestValue></Reference></SignedInfo><SignatureValue>dY9wmKfhW6rqVNdSqoO77JUFQAdDTnsV5016PGU440oDMxwk4B5whA==
</SignatureValue></Signature></Mix></Mixes>
[2003/06/17-22:43:48, debug   ] Try to verify next mix signature...[2003/06/17-22:43:48, debug   ] success!
[2003/06/17-22:43:48, debug   ] CAFirstMix init() succeded
[2003/06/17-22:43:48, debug   ] CAFirstMix InfoService - Signature set
[2003/06/17-22:43:48, debug   ] CAFirstMix Helo sended
[2003/06/17-22:43:48, debug   ] InfoService Loop started
[2003/06/17-22:43:48, debug   ] Starting Message Loop...
\end{verbatim}

When you get that far your Mix is fully operational and you are
ready for the final steps to make your Mix run unattended and start
automatically if your machine should ever need to be rebooted.


% ---------------------------------------------------------------------
% ---------------------------------------------------------------------

\section {Running the Mix unattended}

Rerun ``MixConfig'' to turn proxytest into Daemon and to switch off
logging, or switch to logging via syslog if you want to see it work.
(First and last Mixes will produce a lot of output there.)

% ---------------------------------------------------------------------

\subsection {System Start Scripts}

To make the Mix start automatically in case your system is rebooted
you will need to write a start script. Fortunately most systems
provide skeleton scripts. You only need to adjust them instead of
writing one from scratch.

% ---------------------------------------------------------------------

\subsubsection {Debian}

Become root. Go to /etc/init.d/ and make a copy of the existing
skeleton file. Call it mix.

Start your favorite editor and open that file.

Change
\begin{verbatim}
DAEMON=/usr/sbin/daemon
NAME=daemon
DESC="some daemon"
\end{verbatim}
to
\begin{verbatim}
DAEMON=/home/mix/proxytest
NAME=proxytest
DESC="Mix daemon"
\end{verbatim}

and change

\begin{verbatim}
  start)
        echo -n "Starting $DESC: $NAME"
        start-stop-daemon --start --quiet --pidfile /var/run/$NAME.pid \
                --exec $DAEMON
        echo "."
        ;;
  stop)
        echo -n "Stopping $DESC: $NAME "
        start-stop-daemon --stop --quiet --pidfile /var/run/$NAME.pid \
                --exec $DAEMON
        echo "."
        ;;
\end{verbatim}

to

\begin{verbatim}
  start)
        echo -n "Starting $DESC: $NAME"
        start-stop-daemon --start --quiet -u mix -c mix \
                --exec $DAEMON -- -c /home/mix/config.xml
        echo "."
        ;;
  stop)
        echo -n "Stopping $DESC: $NAME "
        start-stop-daemon --stop --quiet -u mix -c mix \
                --exec $DAEMON
        echo ``.''
        ;;
\end{verbatim}


You can now test your script by running \texttt{/etc/init.d/mix start} and
confirm with \texttt{ps axu} that your mix is running.

Shut it down with \texttt{/etc/init.d/mix stop} and once again
confirm with \texttt{ps axu} that your mix is not running anymore.

If that went OK you can use \texttt{update-rc.d mix defaults 50} to
actually make your script part of the startup and shutdown procedures.

Reboot your machine to finally test it all.

% ---------------------------------------------------------------------

\subsubsection {Suse}

Become root. Go to /etc/init.d/ and make a copy of the existing
skeleton file. Call it mix.

Start your favorite editor, open that file and change the INFO section to something like this:

\begin{verbatim}
### BEGIN INIT INFO
# Provides:          Mix
# Required-Start:    $syslog $network $named
# X-UnitedLinux-Should-Start: $time
# Required-Stop:     $syslog $network
# X-UnitedLinux-Should-Stop: $time
# Default-Start:     3 5
# Default-Stop:      0 1 2 6
# Short-Description: Mix cascade proxytest daemon
# Description:       Start mix to run the proxytest daemon
#       and let it connect to the other mixes in its cascade.
### END INIT INFO
\end{verbatim}

Now change 

\begin{verbatim}
# Check for missing binaries (stale symlinks should not happen)
FOO_BIN=/usr/sbin/FOO
test -x $FOO_BIN || exit 5
\end{verbatim}
to
\begin{verbatim}
# Check for missing binaries (stale symlinks should not happen)
FOO_BIN=/home/mix/proxytest
test -x $FOO_BIN || exit 5
\end{verbatim}

and 

\begin{verbatim}
# Check for existence of needed config file and read it
FOO_CONFIG=/etc/sysconfig/FOO
test -r $FOO_CONFIG || exit 6
. $FOO_CONFIG
\end{verbatim}
to
\begin{verbatim}
# Check for existence of needed config file
FOO_CONFIG=/home/mix/config.xml
test -r $FOO_CONFIG || exit 6
#. $FOO_CONFIG
\end{verbatim}

Further down look for 

\begin{verbatim}
    start)
        echo -n "Starting FOO "
        ## Start daemon with startproc(8). If this fails
        ## the return value is set appropriately by startproc.
        startproc $FOO_BIN

\end{verbatim}

and change that to

\begin{verbatim}
    start)
        echo -n "Starting Mix "
        ## Start daemon with startproc(8). If this fails
        ## the return value is set appropriately by startproc.
        startproc -u mix $FOO_BIN -c $FOO_CONFIG

\end{verbatim}

and for cosmetic reasons also change

\begin{verbatim}
    stop)
        echo -n "Shutting down FOO "
        ## Stop daemon with killproc(8) and if this fails
\end{verbatim}

to

\begin{verbatim}
    stop)
        echo -n "Shutting down Mix "
        ## Stop daemon with killproc(8) and if this fails
\end{verbatim}

You can now test your script by running \texttt{/etc/init.d/mix start} and
confirm with \texttt{ps axu} that your mix is running.

Shut it down with \texttt{/etc/init.d/mix stop} and once again
confirm with \texttt{ps axu} that your mix is not running anymore.

If that went OK you can use \texttt{insserv /etc/init.d/mix} to 
actually make your script part of the startup and shutdown procedures.

Reboot your machine to finally test it all.

% ---------------------------------------------------------------------

\subsubsection{Red Hat}


\verb|export LD_LIBRARY_PATH=/home/mix/xerces-c-src_2_3_0/lib/|


% ---------------------------------------------------------------------

\subsection{Links to further Information}


TO-DO\\

The End.

\end{document}


% LocalWords:  JAP InfoService PGP Debian GPG URL Suse Applet dselect yast Ssh
% LocalWords:  Xerces OpenSSL Netscape Mozilla Konqueror KMail proxytest XML IP
% LocalWords:  item MixConfig Dresden devel JRE
