\documentclass{article}
\setlength{\parindent}{0pt}

\usepackage{url}
\usepackage{hyperref}
\begin{document}

%\newenvironment{myoutput} { \begin{center}\framebox[1.05\width]\{\begin{minipage}[c]{0.8\textwidth} \small\{ \begin{verbatim} } { \end{verbatim}\} \end{minipage} \} \end{center} }

% ---------------------------------------------------------------------
% ---------------------------------------------------------------------


\section{Before You Start}


This document is just a small explaination on how to run proxytest
(\url{http://www.anon-online.de}) on knoppix
(\url{http://www.knopper.net/knoppix/} ).

You should read the mix install manual first.


Things you'll need:

A Knoppix iso image. Downloadable from one of the many mirrors.

The \verb|knoppix-customize| tool also available from these locations.

An USB Memory key with about 10 MB free space.

The \verb|proxytest| binary available from \url{http://www.anon-online.de}

A config.xml file for proxytest.


% ---------------------------------------------------------------------

\subsection{ISO Image preparation}


Download the .iso image and also download the \verb|knoppix-customize| tool.

using \verb|knoppix-customize| add \verb|home=scan| and \verb|myconf=scan|
to the default kernel options.

\begin{verbatim}

echo "home=scan" | knoppix-customize --action set_append_opt --image knoppix.iso 
echo "myconf=scan" | knoppix-customize --action set_append_opt --image knoppix.iso 

\end{verbatim}


% ---------------------------------------------------------------------



\subsection{make start scripts}

boot knoppix. it will look fora home directory but will not find one.

in the .kde/Autostart/ directory create a startmix.sh script to start another
script in a konsole window.


that other script .. lets name it runmix.sh shold be like this

while /bin/true ; do
  ./proxytest -c config.xml
  echo mix exited. restarting in 10 seconds.
  sleep 10
done;

nothing more.

\subsection{preparing the usb key}

insert the key.

it will be recognized imediatly. as "sda1" or something like that.

from the knoppix menue select creating a permanent home directory and follow
the instructions.

you can also save your configuration to the key. including your ip address
in case you dont want to rely on a dhcp server.

now download the proxytest binray from www.anon-online.de and place it in
your home directory (default is ~/Documents).

also place the config.xml that you prepared for the mix in the same
directory.

check that all scripts and binaries have the right permissions (755) and
start your mix-on-cd

if it works you can set the usb key on read only and you are done.

\end{document}


% LocalWords:  JAP InfoService PGP Debian GPG URL Suse Applet dselect yast Ssh
% LocalWords:  Xerces OpenSSL Netscape Mozilla Konqueror KMail proxytest XML IP
% LocalWords:  item MixConfig Dresden devel JRE
